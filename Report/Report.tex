              
%%%%%%%%%%%%%%%%%%%%%%%%%%%%%%%%%%%%%%%%%%%%%%%%%%%%%%%%%%%%%%%%%%%%%
% BY MAHAMDI AMINE
%%%%%%%%%%%%%%%%%%%%%%%%%%%%%%%%%%%%%%%%%%%%%%%%%%%%%%%%%%%%%%%%%%%%%%
\documentclass[12pt]{report}
\usepackage[a4paper]{geometry}
\usepackage[myheadings]{fullpage}
\usepackage{fancyhdr}
\usepackage{lastpage}
\usepackage{graphicx, wrapfig, subcaption, setspace, booktabs}
\usepackage[T1]{fontenc}
\usepackage[font=small, labelfont=bf]{caption}
\usepackage{fourier}
\usepackage[protrusion=true, expansion=true]{microtype}
\usepackage[french,english]{babel}
\usepackage{sectsty}
\usepackage{url, lipsum}
\usepackage[utf8]{inputenc}
\usepackage{indentfirst}
\newcommand{\HRule}[1]{\rule{\linewidth}{#1}}
\onehalfspacing
\setcounter{tocdepth}{5}
\setcounter{secnumdepth}{5}

%-------------------------------------------------------------------------------
% HEADER & FOOTER
%-------------------------------------------------------------------------------
\pagestyle{fancy}
\fancyhf{}
\setlength\headheight{15pt}
\fancyhead[L]{fm\char`_mahamdi@esi.dz    }
\fancyhead[R]{fr\char`boukabene@esi.dz}
\fancyfoot[R]{Page \thepage\ sur \pageref{LastPage}}
%-------------------------------------------------------------------------------
% TITLE PAGE
%-------------------------------------------------------------------------------

\begin{document}
	\renewcommand{\contentsname}{Table des Matières}
	\author{}        
	\date{} 
	\title{  \textsc{ Calcule du nombre de points d'articulation dans un graph non orienter}
		\\ [2.0cm]
		\HRule{0.5pt} \\
		\LARGE \textbf{\uppercase{Rapport de TP n°2 }}
		\HRule{2pt} \\ [0.5cm]
		\normalsize \today \vspace*{5\baselineskip}}
	\maketitle
	\tableofcontents
	\renewcommand{\contentsname}
	\newpage
	%------------------------------------------------------------------------------
	% Section title formatting
	\sectionfont{\scshape}
	
	%------------------------------------------------------------------------------
	% introduction
	%------------------------------------------------------------------------------
	\newpage
	\section*{Introduction}
	 \addcontentsline{toc}{section}{Introduction}
	 \par{}
	 En algorithmique, optimisation combinatoire. Il modélise une situation analogue au remplissage d'un sac à dos, ne pouvant supporter plus d'un certain poids, avec tout ou partie d'un ensemble donné d'objets ayant chacun un poids et une valeur. Les objets mis dans le sac à dos doivent maximiser la valeur totale, sans dépasser le poids ma
	 
		
		
\end{document}
